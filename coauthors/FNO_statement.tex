\documentclass[12pt]{article}
\usepackage{fancyhdr}% http://ctan.org/pkg/fancyhdr
\usepackage{graphicx}
% geometry
\usepackage{geometry}
\geometry{a4paper,inner=15mm,outer=12mm,top=12.5mm,bottom=15mm,footskip=5mm}%
% fancyheader
\pagestyle{fancy}% Change page style to fancy
\fancyhf{}% Clear header/footer
\rhead{%
\includegraphics[width=.3\textwidth]{../atelier/FAUWortmarkeBlau.pdf}%
}
\lhead{%
	\includegraphics[width=.5\textwidth]{../atelier/NatKennung.pdf}%
}
%\fancyfoot[C]{Footer}% \fancyfoot[R]{\thepage}
\renewcommand{\headrulewidth}{0pt}
\setlength{\headheight}{70pt} 
\begin{document}
\section*{Bestätigung der Beiträge in Publikationen}

Dieses Dokument bestätigt nach \S10 Absatz (3) Satz 4 der Fakultätspromotionsordnung für die Naturwissenschaftliche Fakultät der
Friedrich-Alexander-Universität Erlangen-Nürnberg (FPromO Nat) vom 21. Januar 2013, die Urheberschaft einer für eine Dissertation relevanten Arbeit.\par

\vspace{15pt}
\noindent
Hiermit wird bestätigt, dass der Beitrag des Doktoranden \textbf{Tim Roith} in der Publikation
%
\begin{center}
S. Kabri, T. Roith, D. Tenbrinck, and M. Burger. “Resolution-Invariant
Image Classification based on Fourier Neural Operators.” In: Scale Space
and Variational Methods in Computer Vision: 9th International Conference,
SSVM 2023, Proceedings. Springer. 2023, pp. 307–319,
\end{center}
%
wie in den nachfolgenden Erklärungen beschrieben, gegeben ist.

\paragraph{Deutsche Version:} Diese Arbeit beruht auf SKs Masterarbeit und verwendet die ursprünglichen Ideen MBs, zu Auflösungsinvarianz mithilfe von FNOs. Im Paper erarbeitete TR die Beweise zur Wohldefiniertheit und Fréchet-Differenzierbarkeit, zusammen mit SK. Er schrieb große Teile des Papers und des Source-Codes, wobei DT bei der Korrektur der publizierten Version mitgeholfen hat. Hierbei führte er die numerischen Studien zusammen mit SK durch. 

\paragraph{Englische Version:} This work is based on SK's masters thesis, employing the initial ideas of MB for resolution invariance with FNOs. In the paper TR worked out the proofs for well-definedness and Fréchet-differentiability, together with SK. He wrote large parts of the paper and the source code, where DT helped with proofreading of the published version. Here, he conducted the numerical studies in collaboration with SK.
\vspace{50pt}

\renewcommand{\arraystretch}{4}
\begin{tabular}{l l l}
Name & Datum & Unterschrift\\
Samira Kabri & \raisebox{-3pt}{\makebox[5cm]{.\dotfill}} & \raisebox{-3pt}{\makebox[5cm]{.\dotfill}}\\
%
Daniel Tenbrinck & \raisebox{-3pt}{\makebox[5cm]{.\dotfill}} & \raisebox{-3pt}{\makebox[5cm]{.\dotfill}}\\
Martin Burger & \raisebox{-3pt}{\makebox[5cm]{.\dotfill}} & \raisebox{-3pt}{\makebox[5cm]{.\dotfill}}\\
\end{tabular}%
%
\vspace{50pt}

Der Betreuer \textbf{Martin Burger} hat diese Erklärung als Ko-Author zur Kenntnis genommen.

 

\end{document}