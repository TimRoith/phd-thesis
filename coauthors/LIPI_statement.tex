\documentclass[12pt]{article}
\usepackage{fancyhdr}% http://ctan.org/pkg/fancyhdr
\usepackage{graphicx}
% geometry
\usepackage{geometry}
\geometry{a4paper,inner=15mm,outer=12mm,top=12.5mm,bottom=15mm,footskip=5mm}%
% fancyheader
\pagestyle{fancy}% Change page style to fancy
\fancyhf{}% Clear header/footer
\rhead{%
\includegraphics[width=.3\textwidth]{../atelier/FAUWortmarkeBlau.pdf}%
}
\lhead{%
	\includegraphics[width=.5\textwidth]{../atelier/NatKennung.pdf}%
}
%\fancyfoot[C]{Footer}% \fancyfoot[R]{\thepage}
\renewcommand{\headrulewidth}{0pt}
\setlength{\headheight}{70pt} 
\begin{document}
\section*{Bestätigung der Beiträge in Publikationen}

Dieses Dokument bestätigt nach \S10 Absatz (3) Satz 4 der Fakultätspromotionsordnung für die Naturwissenschaftliche Fakultät der
Friedrich-Alexander-Universität Erlangen-Nürnberg (FPromO Nat) vom 21. Januar 2013, die Urheberschaft einer für eine Dissertation relevanten Arbeit.\par

\vspace{15pt}
\noindent
Hiermit wird bestätigt, dass der Beitrag des Doktoranden \textbf{Tim Roith} in der Publikation
%
\begin{center}
T. Roith and L. Bungert. “Continuum limit of Lipschitz learning on graphs.”
In: Foundations of Computational Mathematics (2022), pp. 1–39.
\end{center}
%
wie in den nachfolgenden Erklärungen beschrieben, gegeben ist.

\paragraph{Deutsche Version:} Diese Arbeit baut auf den Erkenntnissen von TRs Masterarbeit auf. Es ist allerdings wichtig anzumerken, dass die Resultate signifikant erweitert wurden und konzeptionell stärker als die der Masterarbeit sind, siehe dazu Abschnitt 3.3 in der Dissertation. TR adaptierte die Continuum Limit Theorie für den $L^\infty$-Fall, erarbeitet die meisten Beweise und schrieb einen großen Teil des Papers. In Zusammenarbeit mit LB, identifizierte er entscheidende Gebiets-Annahmen, welche es erlauben auch mit nicht-konvexen Gebieten zu arbeiten und bewies Konvergenz für angenäherte Randbedingungen.  

\paragraph{Englische Version:} This work builds upon the findings in TR's masters thesis. It is however important to note that the results constitute a significant extension and are conceptually stronger than the ones in the masters thesis, see Section 3.3 of the dissertation. TR adapted the continuum limit framework to the $L^\infty$ case, worked out most of the proofs and wrote a significant part of the paper. In collaboration with LB, he identified the crucial domain assumptions that allow to work on non-convex domains and proved convergence for approximate boundary conditions.
\vspace{50pt}

\renewcommand{\arraystretch}{4}
\begin{tabular}{l l l}
Name & Datum & Unterschrift\\
Leon Bungert & \raisebox{-3pt}{\makebox[5cm]{.\dotfill}} & \raisebox{-3pt}{\makebox[5cm]{.\dotfill}}\\
%
\end{tabular}%
%
\vspace{50pt}

\noindent
Der Betreuer \textbf{Martin Burger} hat diese Erklärung zur Kenntnis genommen.

\begin{tabular}{l l l}
	Martin Burger & \raisebox{-3pt}{\makebox[5cm]{.\dotfill}} & \raisebox{-3pt}{\makebox[5cm]{.\dotfill}}
	%
\end{tabular}%

 

\end{document}