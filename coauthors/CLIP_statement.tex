\documentclass[12pt]{article}
\usepackage{fancyhdr}% http://ctan.org/pkg/fancyhdr
\usepackage{graphicx}
% geometry
\usepackage{geometry}
\geometry{a4paper,inner=15mm,outer=12mm,top=12.5mm,bottom=15mm,footskip=5mm}%
% fancyheader
\pagestyle{fancy}% Change page style to fancy
\fancyhf{}% Clear header/footer
\rhead{%
\includegraphics[width=.3\textwidth]{../atelier/FAUWortmarkeBlau.pdf}%
}
\lhead{%
	\includegraphics[width=.5\textwidth]{../atelier/NatKennung.pdf}%
}
%\fancyfoot[C]{Footer}% \fancyfoot[R]{\thepage}
\renewcommand{\headrulewidth}{0pt}
\setlength{\headheight}{70pt} 
\begin{document}
\section*{Bestätigung der Beiträge in Publikationen}

Dieses Dokument bestätigt nach \S10 Absatz (3) Satz 4 der Fakultätspromotionsordnung für die Naturwissenschaftliche Fakultät der
Friedrich-Alexander-Universität Erlangen-Nürnberg (FPromO Nat) vom 21. Januar 2013, die Urheberschaft einer für eine Dissertation relevanten Arbeit.\par

\vspace{15pt}
\noindent
Hiermit wird bestätigt, dass der Beitrag des Doktoranden \textbf{Tim Roith} in der Publikation
%
\begin{center}
L. Bungert, R. Raab, T. Roith, L. Schwinn, and D. Tenbrinck. “CLIP:
Cheap Lipschitz training of neural networks.” In: Scale Space and Variational
Methods in Computer Vision: 8th International Conference, SSVM
2021, Proceedings. Springer. 2021, pp. 307–319,
\end{center}
%
wie in den nachfolgenden Erklärungen beschrieben, gegeben ist.

\paragraph{Deutsche Version:} TR erarbeitete den Algorithmus der im Paper vorgeschlagen wird zusammen mit LB, basierend auf LBs Idee. Zusammen mit LS, RR und DT führte er die numerischen Beispiele durch und schrieb große Teile des Quellcodes. Weiterhin schrieb er entscheidende Teile des Paper, wobei DT das Dokument Korrektur lies und klarer formulierte.

\paragraph{Englische Version:} TR worked out the main algorithm proposed in the paper together with LB, based on LB's idea. Together with LS, RR and DT he conducted the numerical examples and also wrote large parts of the source code. Furthermore, he wrote significant parts of the paper, where DT proofread and clarified the final document.
\vspace{30pt}

\renewcommand{\arraystretch}{3.5}
\begin{tabular}{l l l}
Name & Datum & Unterschrift\\
Leon Bungert & \raisebox{-3pt}{\makebox[5cm]{.\dotfill}} & \raisebox{-3pt}{\makebox[5cm]{.\dotfill}}\\
René Raab & \raisebox{-3pt}{\makebox[5cm]{.\dotfill}} & \raisebox{-3pt}{\makebox[5cm]{.\dotfill}}\\
%
Leo Schwinn & \raisebox{-3pt}{\makebox[5cm]{.\dotfill}} & \raisebox{-3pt}{\makebox[5cm]{.\dotfill}}\\
Daniel Tenbrinck & \raisebox{-3pt}{\makebox[5cm]{.\dotfill}} & \raisebox{-3pt}{\makebox[5cm]{.\dotfill}}\\
\end{tabular}%
%

\vspace{30pt}
\noindent
Der Betreuer \textbf{Martin Burger} hat diese Erklärung zur Kenntnis genommen.

\begin{tabular}{l l l}
	Martin Burger & \raisebox{-3pt}{\makebox[5cm]{.\dotfill}} & \raisebox{-3pt}{\makebox[5cm]{.\dotfill}}
	%
\end{tabular}%

 

\end{document}