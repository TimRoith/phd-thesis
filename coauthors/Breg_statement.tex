\documentclass[12pt]{article}
\usepackage{fancyhdr}% http://ctan.org/pkg/fancyhdr
\usepackage{graphicx}
% geometry
\usepackage{geometry}
\geometry{a4paper,inner=15mm,outer=12mm,top=12.5mm,bottom=15mm,footskip=5mm}%
% fancyheader
\pagestyle{fancy}% Change page style to fancy
\fancyhf{}% Clear header/footer
\rhead{%
\includegraphics[width=.3\textwidth]{../atelier/FAUWortmarkeBlau.pdf}%
}
\lhead{%
	\includegraphics[width=.5\textwidth]{../atelier/NatKennung.pdf}%
}
%\fancyfoot[C]{Footer}% \fancyfoot[R]{\thepage}
\renewcommand{\headrulewidth}{0pt}
\setlength{\headheight}{70pt} 
\begin{document}
\section*{Bestätigung der Beiträge in Publikationen}

Dieses Dokument bestätigt nach \S10 Absatz (3) Satz 4 der Fakultätspromotionsordnung für die Naturwissenschaftliche Fakultät der
Friedrich-Alexander-Universität Erlangen-Nürnberg (FPromO Nat) vom 21. Januar 2013, die Urheberschaft einer für eine Dissertation relevanten Arbeit.\par

\vspace{15pt}
\noindent
Hiermit wird bestätigt, dass der Beitrag des Doktoranden \textbf{Tim Roith} in der Publikation
%
\begin{center}
L. Bungert, T. Roith, D. Tenbrinck, and M. Burger. “A Bregman learning
framework for sparse neural networks.” In: Journal of Machine Learning
Research 23.192 (2022), pp. 1–43,
\end{center}
%
wie in den nachfolgenden Erklärungen beschrieben, gegeben ist.

\paragraph{Deutsche Version:} TR erweiterte LBs Idee, Bregman Iterationen für sparses Training einzusetzen, konzipiert durch DT. Zusammen mit MB und LB erarbeitete er die Konvergenzbeweise der stochastischen Bregman Iteration. Hier schlug er auch eine fundierte sparse Initialisierungsstrategie vor. Weiterhin führte er die numerischen Beispiele durch und schrieb den größten Teil des Quellcodes.

\paragraph{Englische Version:} TR expanded LB's ideas of employing Bregman iteration for sparse training, conceptualized by DT. Together with MB and LB he worked out the convergence analysis of stochastic Bregman iterations. Here, he also proposed a profound sparse initialization strategy. Furthermore, he conducted the numerical examples and wrote most of the source code.
\vspace{50pt}

\renewcommand{\arraystretch}{4}
\begin{tabular}{l l l}
Name & Datum & Unterschrift\\
Leon Bungert & \raisebox{-3pt}{\makebox[5cm]{.\dotfill}} & \raisebox{-3pt}{\makebox[5cm]{.\dotfill}}\\
%
Daniel Tenbrinck & \raisebox{-3pt}{\makebox[5cm]{.\dotfill}} & \raisebox{-3pt}{\makebox[5cm]{.\dotfill}}\\
Martin Burger & \raisebox{-3pt}{\makebox[5cm]{.\dotfill}} & \raisebox{-3pt}{\makebox[5cm]{.\dotfill}}\\
\end{tabular}%
%
\vspace{50pt}

Der Betreuer \textbf{Martin Burger} hat diese Erklärung als Ko-Autor zur Kenntnis genommen.

 

\end{document}