\documentclass[12pt]{article}
\usepackage{fancyhdr}% http://ctan.org/pkg/fancyhdr
\usepackage{graphicx}
% geometry
\usepackage{geometry}
\geometry{a4paper,inner=15mm,outer=12mm,top=12.5mm,bottom=15mm,footskip=5mm}%
% fancyheader
\pagestyle{fancy}% Change page style to fancy
\fancyhf{}% Clear header/footer
\rhead{%
\includegraphics[width=.3\textwidth]{../atelier/FAUWortmarkeBlau.pdf}%
}
\lhead{%
	\includegraphics[width=.5\textwidth]{../atelier/NatKennung.pdf}%
}
%\fancyfoot[C]{Footer}% \fancyfoot[R]{\thepage}
\renewcommand{\headrulewidth}{0pt}
\setlength{\headheight}{70pt} 
\begin{document}
\section*{Bestätigung der Beiträge in Publikationen}

Dieses Dokument bestätigt nach \S10 Absatz (3) Satz 4 der Fakultätspromotionsordnung für die Naturwissenschaftliche Fakultät der
Friedrich-Alexander-Universität Erlangen-Nürnberg (FPromO Nat) vom 21. Januar 2013, die Urheberschaft einer für eine Dissertation relevanten Arbeit.\par

\vspace{15pt}
\noindent
Hiermit wird bestätigt, dass der Beitrag des Doktoranden \textbf{Tim Roith} in der Publikation
%
\begin{center}
L. Bungert, J. Calder, and T. Roith. “Uniform convergence rates for Lipschitz
learning on graphs.” In: IMA Journal of Numerical Analysis (Sept.
2022),
\end{center}
%
wie in den nachfolgenden Erklärungen beschrieben, gegeben ist.

\paragraph{Deutsche Version:} In Zusammenarbeit mit LB, arbeitete TR an den Konvergenzbeweisen, basierenden auf den Ideen von JC. Zusammen mit LB und JC bewies er das Hauptresultat und die verschiedenen Lemmas die darauf hinführen. Hierbei beschäftigte er sich vor allem mit der Adaption der Theorie für AMLEs auf den Graph-Fall, was das entscheidende Element für die ganze Arbeit ist. Weiterhin, trug er zur Gestaltung und Implementierung der numerischen Experimente, die im Paper durchgeführt wurden bei. 

\paragraph{Englische Version:} In collaboration with LB, TR worked on the convergence proofs building upon the ideas of JC. Together with LB and JC he proved the main convergence result and the various lemmas leading up to it. Here, he was especially concerned with the adaptation of the theory for AMLEs to the graph case, with is a crucial element for the whole work. Furthermore, he contributed to the design and implementation of the numerical examples conducted in the paper.
\vspace{50pt}

\renewcommand{\arraystretch}{4}
\begin{tabular}{l l l}
Name & Datum & Unterschrift\\
Leon Bungert & \raisebox{-3pt}{\makebox[5cm]{.\dotfill}} & \raisebox{-3pt}{\makebox[5cm]{.\dotfill}}\\
%
Jeff Calder & \raisebox{-3pt}{\makebox[5cm]{.\dotfill}} & \raisebox{-3pt}{\makebox[5cm]{.\dotfill}}\\
\end{tabular}%
%
\vspace{50pt}

\noindent
Der Betreuer \textbf{Martin Burger} hat diese Erklärung zur Kenntnis genommen.

\begin{tabular}{l l l}
	Martin Burger & \raisebox{-3pt}{\makebox[5cm]{.\dotfill}} & \raisebox{-3pt}{\makebox[5cm]{.\dotfill}}
	%
\end{tabular}%

 

\end{document}